\chapter{ACKNOWLEDGEMENTS}

\begin{singlespace}

\noindent

%% \textit{Bismillahir-Rahmanir-Rahim. Innal hamdulillah, wa nahmaduhu, wa nasta'ienuhu, wa nastaghfiruhu. Wa na'udzubihi, min syururi anfusina, wa min sayyita a'qmalina. Man yahdhihillahu, fala mudhillalah. Wa man yudhlil, fala ha diyalah. Assalamualaikum Warahmatullah Hiwabarokatuh.}

\noindent	
In the name of Allah, Most Gracious and Most Merciful. Indeed all praises be unto Allah, and we praise Him, and we seek help from Him, and we seek forgiveness from Him. We seek refuge from Him from the evil of ourselves, and from the evil of our actions. Whomsoever Allah guides, none can misguide. And whomsoever He leaves astray, none can guide. May the Peace, Mercy and Blessings of Allah be upon you.\\  

\noindent
Foremost, I thank Allah, the Most Glorious and Most High, for granting me the opportunity to tread down the unknown trail on this research journey. I wish to also convey my sincerest gratitude to all people who have directly or indirectly, or will be involved with me on this journey. There are just too many people to mention.   \\

\noindent
Specially to Prof. Yashwant Prasad Singh, perceived as many personalities to me: As my guru, teacher, mentor, advisor, supervisor and a dear friend, I am very grateful to him for his unimaginable faith, persistence and enthusiasm in encouraging, guiding, and sharing with me knowledge throughout the many wonderful years of our academic acquaintance. As my Supervisor, his advice was simple, "Your PhD study should be exciting and fun." And that is true. We spent long hours and fruitful discussions on almost unlimited topics, from philosophy, politics, religion, and family to the hard sciences, computer science, and engineering. We also made a pact to remain as lifelong friends, Insya Allah, God Willing. With internet facilities today, we are constantly in communication.\\

\noindent
As a tribute to Assoc. Prof. Dr. Fadhlur Rahman, my direct supervisor, I am eternally indebted to him for his sincere trust, unbelievable patience, constant guidance and timely assistance with all matters, including research equipment, resources and many other administrative needs of the university. To my brother, Prof. Ir. Dr. Wan Azhar, I undyingly appreciate his challenge that I should eventually get a doctorate. To my son, Abdulazeez, I thank him adoringly for his unequivocal faith, continuous encouragement and financial assistance in my endeavor.\\

\noindent
To my family, especially my wife, my sons and daughter, I thank them affectionately for their unshakable love, utmost patience, undivided support and unwavering understanding during the long hours and sleepless nights I went through. The coffee and snacks were never interrupted. For smooth English writing, my wife is also my bouncer and editor. \\

\noindent
To those in my extended family with PhDs, I thank them admiringly for their strange looks and jokes. One senior poked fun with a comment at me. He said, "Considered to be the most intelligent among us, he does not have a doctorate. Hahaha." My cynical response was, "Is it not inspiring that for many years I have been doing work of people with PhDs, but without a doctorate myself? I feel, it is certainly humbling but yet assuring being accorded that level of believe in my ability."\\

\noindent
To my friends, I kindly thank them for all support and encouragement rendered to me during my research journey. To Multimedia University (MMU), I thank them for the opportunity provided to me for teaching, research and those unforgettable interactions with Prof Singh and staff at the university. To University Malaysia Pahang (UMP), I thank them for accepting me as a research student (at the age of mid 60s) and for the generosity of providing research equipment and resources.\\

\noindent
Finally, I again praise Allah, for the invaluable gifts of health, time and clear state of mind, without which, I would not have been able to go through this arduous and exhausting journey. There is always a purpose in everything. Thinking of it, I recalled the motto of a local university, "To Allah and Mankind". Without hesitation, I say, this is exactly the one for me. God Willing, may Allah grant me this deserving wish. In closing, I wish to share the following passages from Allah, the All-Knowing and All-Mighty. \\

\noindent
\textit{And in His Providence are the keys of the Unseen; none knows them except He. And He knows whatever is in the land and the sea. And in no way does a leaf fall down, except that He knows it, and not a grain in the darkness of the earth, not a thing wet or dry, except that it is in an evident Book. Whoever submits his whole self to God and is a doer of good, he has grasped indeed the most trustworthy handhold. And with God rests the end and decision of all affairs. Verily, When He (Allah) intends a thing, his command is "Be ! and it is !".}  \\

\noindent
(\textit{Combined verses Al-Quran, Chapters Al-An'am 6:59, Lukman 31:22 and Ya-Sin 36:82})\\
	
%% (\textit{Al-Quran, Al-An'am 6:59, Lukman 31:22 and Ya-Sin 36:82})

\noindent
A recruiter once told this famous story. He was invited to attend a college team competition where autonomous underwater vehicles must avoid obstacles before reaching the finish line. The competition was held in a swimming pool. After the competition, seeing that the runner up team was sad, he approached them and asked a question, "How much did you spend on your project?" The answer he got was 20K. He remarked, "Oh yeah. But the champion spent 100K. Now who is better?" \\

\noindent
Alhamdulillah. With all my love. Always.\\

\noindent
Wan Ruslan Yusoff\\
UMP Pekan, Pahang\\
\today \\
\end{singlespace}

