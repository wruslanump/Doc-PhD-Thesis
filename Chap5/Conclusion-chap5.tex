\chapter{CONCLUSION}

\section{Conclusions of the research}

The realtime interpolation algorithm developed in this work, when executed against all of the selected curves exclusively and simultaneously satisfy both the design constraints on the feedrate and its chord-error tolerance. The summary of work achievements are as follows.

\begin{enumerate}
	\item The resulting feedrate profiles throughout the entire path are continuous and smooth. 
	
	\item The maximum feedrate at every interpolated point remains below and close to the calculated feedrate limit. 
	
	\item The current running feedrate does not exceed the user specified command feedrate at every point during the traversal along the entire parametric curve. 
	
	\item The chord-error at every interpolated arc segment remains below the chord-error tolerance throughout the curve traversal. 
	
	\item Without fail, a single and robust interpolation algorithm was developed and is capable of handling the complexities of a wide range of 2D parametric curves comprising different sizes and shapes. 
	
	\item The algorithm can be implemented both in a realtime, online mode by driving the CNC machine directly, or in an offline mode by using a stored RS274/NGC G-code standard file. This work satisfies both interpretations of realtime: the lay person interpretation and the technical definition of realtime. 
	
	\item The performance accuracy of the algorithm for the chord-error per unit length traversed and error-tolerance set at $(10)^{-6}$ mm, as expected, varies for different curves. 
	
	\item The average performance of chord-error to length ratio is in the range of $(10)^{-6}$ to $(10)^{-5}$ for the algorithm. The results are provided in [Table \ref{Chord-error-per-unit-curve-traversed-FC20}]. 

\end{enumerate}	

	
\section{Performance accuracy of the algorithm}	
	
\noindent
The curve-error per unit length traversed is the sum total of all chord-errors divided by the sum total of all chord lengths for each parametric curve. This ratio is a performance accuracy measure of the interpolation algorithm developed in this work.
	
\begin{table}[ht]
%%	\begin{center}
\caption{Chord-error per unit length curve traversed (ratio) CEE}
\label{Chord-error-per-unit-curve-traversed-FC20}
\begin{tabular}{p{0.5cm} p{4.0cm} p{2.5cm} p{2.0cm} p{2.0cm} p{2.0cm} }
\hline	
			  &              &                & CEE    & CEE   & CEE \\
			  &              &    Total Chord & Ratio & Ratio & Ratio \\
			  & Curve Type   &    Length(mm)  & Min   &  Avg  & Max\\
			\hline
			1 &	Teardrop curve      & 101.83  & $5.70(10)^{-5}$ & $6.45(10)^{-5}$ & $7.20(10)^{-5}$ \\ 
			2 &	Butterfly curve     & 356.07  & $5.45(10)^{-6}$ & $1.09(10)^{-5}$ & $1.64(10)^{-5}$ \\ 
			3 &	Ellipse curve       & 215.64  & $1.38(10)^{-5}$ & $2.39(10)^{-5}$ & $3.40(10)^{-5}$ \\ 
			4 &	SkewedAstroid curve & 445.71  & $1.16(10)^{-6}$ & $2.89(10)^{-5}$ & $4.62(10)^{-6}$ \\ 
			5 &	Circle curve        & 496.39  & $2.20(10)^{-6}$ & $4.75(10)^{-6}$ & $7.30(10)^{-6}$ \\
			6 &	AstEpi curve        & 426.26  & $1.97(10)^{-6}$ & $5.39(10)^{-6}$ & $8.81(10)^{-6}$ \\  
			7 &	Snailshell curve    & 138.56  & $3.69(10)^{-5}$ & $4.38(10)^{-5}$ & $5.08(10)^{-5}$ \\
			8 &	SnaHyp curve        & 478.99  & $5.94(10)^{-6}$ & $7.62(10)^{-6}$ & $9.31(10)^{-6}$ \\
			9 &	Ribbon10L curve     & 15.21   & $4.81(10)^{-4}$ & $4.82(10)^{-4}$ & $4.82(10)^{-4}$ \\ 
			10 & Ribbon100L curve   & 152.13  & $4.75(10)^{-5}$ & $4.79(10)^{-5}$ & $4.82(10)^{-5}$ \\ 	
\hline
\end{tabular}
%%	\end{center}
\end{table}

\noindent
The ratio in the table above is termed Chord-Error-Efficiency (CEE) and interpreted as the amount of error generated per unit chord-length of curve travel. This CEE ratio is independent of the total length of the curve. \\

Ignoring the Ribbon10L curve for being too small (4 mm x 4 mm), the average performance of chord-error to length ratio is in the range of $(10)^{-6}$ to $(10)^{-5}$ for the algorithm developed in this work. 


		
%% ==============================================
\clearpage
\pagebreak

\section{Knowledge contributions}

\begin{enumerate}
	\item A parametric curve interpolation method that successfully constrain chord-error and feedrate simultaneously with confinement of tangential acceleration.
	
	\item The importance of having a fully functionalized and modularized implementation code for interpolation that follows good software engineering practice.   
	
\end{enumerate}


\section{Recommendations for future work}

It is highly recommended that future interpolation work involves NURBS curves. The availability of libraries in various programming languages to manipulate NURBS, like C/C++, Python, Julia and Octave, opens up possibilities for commercial applications of NURBS in CNC machining. It is recommended that a study be conducted to compare the simplicity, accuracy and intensiveness of computations using NURBS libraries of the different programming languages.  





%% =============================
\cleardoublepage

