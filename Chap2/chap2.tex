\chapter{LITERATURE SURVEY}

\section{Introduction}

By default {\LaTeX} has its own standard format. This include paragraph alignment, paragraph indents, line spacing, abstracts, references and appendixes.  To comply with UMP thesis standards we have developed a custom coding to the preamble file that enable {\LaTeX} to produce  document according to the UMP standard formatting. This will make sure a uniform and error free format of your thesis. This chapter will cover techniques to format material in the document you create.

\section{Sectioning command}

For producing large documents such as thesis it is more easier to split the content into parts using {\LaTeX} sectioning commands. {\LaTeX} provides 7 levels of depth for defining sections as in Table \ref{tab1:section_depth}. For getting started, start each document with \verb+\chapter{title}+. This will make sure your thesis is well organized. Notice that you do not need to specify section numbers; LaTeX will sort that out for you. In addition, all the titles of the sections are added automatically to the table of contents. Moreover, the link to each chapter, section, tables, figures etc. are also provided automatically by {\LaTeX}.

% Table generated by Excel2LaTeX from sheet 'Sheet1'

\begin{table}[htbp]
\caption{Comparison of \LaTeX-based or \LaTeX-supporting Equation Editors} 
\centering
\begin{tabularx}{1\linewidth}{lll}    \addlinespace
    \toprule
    Command & Level & Comment \\
    \toprule
    \verb+\part{part}+ & -1    & not in letters \\
    \verb+\chapter{section}+ & 0     & only books and reports \\
    \verb+\section{section}+ & 1     & not in letters \\
    \verb+\subsection{subsection}+ & 2     & not in letters \\
    \verb+\subsubsection{subsubsection}+ & 3     & not in letters \\
    \verb+\paragraph{paragraph}+ & 4     & not in letters \\
    \verb+\subparagraph{subparagraph}+ & 5     & not in letters \\
    \bottomrule
    \end{tabularx}%
  \label{tab1:section_depth}%
\end{table}%

\section{Font Sizes and Styles}

To change the font size, use any one of the following commands. To change it for just a portion of the page, enclose that potion in \verb+{ }+ and have the relevant font size command occur right at the beginning of the text inside the curly braces. In order from smallest to largest, the font sizes you can use are:
\\
{\tiny{\verb+\tiny+}} \\
{\scriptsize{\verb+\scriptsize+}} \\
{\footnotesize{\verb+\footnotesize+}} \\
{\small{\verb+\small+}} \\
{\normalsize{\verb+\normalsize+}} \\
{\large{\verb+\large+}} \\
{\Large{\verb+\Large+}} \\
{\LARGE{\verb+\LARGE+}} \\
{\huge{\verb+\huge+}} \\ 
{\Huge{\verb+\Huge+}} \\

Try this out; the effects should be pretty clear:\\

When I was born, I was {\small small}. Actually, {\scriptsize I was very small}. When I got older, I thought some day {\Large I would be large}, {\Huge maybe even gigantic}. But instead, I'm not even normalsize. {\small I'm still small.}\\

Here is a simple example that will probably show you all you need to know about bold, italics, and underlining.\\

When something is \emph{really}, \textbf{really} important, you can \underline{underline it}, \emph{italicize it}, \textbf{bold it}. If you \underline{\textbf{\emph{must do all three}}}, then you can nest them.\\

Here is another example that demonstrates font families:\\

You may want to write things \textsf{in a sans-serif font}, or \texttt{in a typewriter font}, or \textsl{in a slanted font} (which is \emph{slightly different} than italics). Sometimes it pays \textsc{to write things in small capitals}. You can next go to \textbf{bold and then \textsl{bold and slanted} and then back to just bold} again.

\section{Spacing}

There are a few spacing items you'll find useful in {\LaTeX}. First, you can force a normal-size space (as between words) by using a single backslash followed by a space. This is particularly useful after periods: LaTeX interprets periods as ends of sentences, so it puts extra space after them, but if a period doesn't in fact end a sentence, you don't want that extra space. Try this to see an example.\\

When Mr. Rogers read this, he was confused because the first sentence was only two words long. Mrs.\ Rogers wasn't confused at all.\\

In math mode, it's a little different. LaTeX ignores normal spaces in math mode, so all three of the following will come out the same:\\

Spacing in math mode:

$x + y$

$x     +       y$

$x+y$
\\
Notice that the three math expressions come out all exactly the same. In general, you can trust math mode to space things out right rather than forcing any special spacing. This means that you should write formulas in your source document to be easily readable (by you), and trust LaTeX to do the right spacing.\\

However, if you do need to tweak the spacing in math mode, there are some special commands:
\, 	a small space
\: 	a medium space
\; 	a large space
\quad 	a really large space
\qquad 	a huge space
\! 	a negative space (moves things back to the left)\\

Here are examples of these in action:

$x+y$

$x+\,y$

$x+\:y$

$x+\;y$

$x+\quad y$

$x+\qquad y$

$x+\!y$

\subsection{Justification (Centering, etc)}

We can center text using
\\
\\
\verb+\begin{center}+\\
text\\
\verb+\end{center}+
\\
\\
and can justify it left or right with
\\
\\
\verb+\begin{flushleft}+\\
text\\
\verb+\end{flushleft}+\\
\\
\verb+\begin{flushright}+\\
text\\
\verb+\end{flushright}+\\
\\
\begin{center}
Fourscore and seven years ago our fathers brought forth on this continent a new nation, conceived in liberty and dedicated to the
proposition that all men are created equal.
\end{center}

\begin{flushleft}
Now we are engaged in a great civil war, testing whether that nation or any nation so conceived and so dedicated can long endure. We are met on a great battlefield of that war. We have come to dedicate a portion of it as a final resting place for those who died here that the nation might live. This we may, in all propriety do. But in a larger sense, we cannot dedicate, we cannot consecrate, we cannot hallow this ground. The brave men, living and dead who struggled here have hallowed it far above our poor power to add or detract. The world will little note nor long remember what we say here, but it can never forget what they did here.
\end{flushleft}

\begin{flushright}
It is rather for us the living, we here be dedicated to the great task remaining before us--that from these honored dead we take
increased devotion to that cause for which they here gave the last full measure of devotion--that we here highly resolve that these dead shall not have died in vain, that this nation shall have a new birth of freedom, and that government of the people, by the people, for the people shall not perish from the earth.
\end{flushright}

